\chapter{Preface}

\begin{shaded}
\noindent
\textbf{Status as of 2016-10-28.} We have updated the manuscript to track package updates since the previous version uploaded nearly three months ago, and added examples of the new functionality added to packages \pkg{ggspectra}, \pkg{ggrepel}, and \pkg{ggplot2}. Now seven of the packages in the suite are in CRAN. Package \pkg{photobiology} has gone through a major update of the astronomy-related functions. The user interface has changed a little. The values returned are slightly different as a different algorithm has been implemented. Package \pkg{photobiologyInOut} has been expanded in its scope. Bugs have been fixed, but most of them only affected borderline cases.

Some errors in the text of the manuscript have been corrected. During the last couple of months more time was spent in trying to get all the packages in the suite ready for submission to CRAN than on expanding and revising the text of this book. However, quite many of the code examples in the book have been simplified or updated to make use of all the improvements to the packages. Many new plotting examples were added.

I expect the next significant update to be ready for uploading to Leanpub in one or two months time. 
\end{shaded}

This handbook describes how to use R as a tool for doing calculations related to research in photobiology. Photobiology is the branch of science that studies the interactions of living organisms with visible and ultraviolet radiation. Many of the most frequently used calculations are either related to the characterization of radiation and of the responses of organisms to radiation. We emphasize the first of these aspects as in many cases the characterization of the responses of organisms to radiation differ little from the characterization of similar responses elicited through other physical or chemical stimuli.
Many of the examples in this handbook make use of the `r4photobiology'  suite of R packages, but the use of other packages is also described.

The biophysical theory needed to understand the purpose and use of the different calculations is presented in the chapters of Part~\ref{part:theory} (p.~\pageref{part:theory} ss.). This first part is simple and concise, and full understanding of the subject matter will require previous experience or further reading \autocite[e.g.][]{Aphalo2012,Bjoern2015}.

The software used in the examples including a suite of R packages developed by one of the authors is described in Part~~\ref{part:tools} (p.~\pageref{part:tools} ss.). However, as several up-to-date texts suitable for learning the R language are available, we assume that readers either already have experience with R, or will familiarize with R using other books \autocite[e.g.][]{Aphalo2016,Horton2015a,Paradis2005,Peng2016}.

A cookbook with recipes for different calculation tasks is included as Part~\ref{part:cookbook} (p.~\pageref{part:cookbook} ss.). These tasks cover a wide range of subjects but emphasis is on spectral data and their manipulation and how on to obtain from them different summary quantities relevant to different organisms. These calculations include in many cases weighted spectral data. Calculations of day length, times of sunset and sunrise, twilight, and the position of the sun are also described. Finally calculations related to colour vision are briefly described.

Part~\ref{part:foreign} (p.~\pageref{part:foreign} ss.) includes chapters dedicated to reading and writing data in \emph{foreign} formats, including direct acquisition from spectrometer and other measuring instruments. Part~\ref{part:data} (p.~\pageref{part:data} ss.) gives a brief overview of the data sets included in the suite.

Although this handbook includes many different recipes, it is not comprehensive in covering all the functionality of the packages. The packages themselves include \emph{User Guides} and help pages describing their functionality in detail. The documentation is available as a web site at \url{http://docs.r4photobiology.info}. A series of articles describing specific aspects of the use of the suite is being published in the \href{http://uv4plants.org/publications/uv4plants-bulletin-published-issues/}{UV4Plants Bulletin}.

This handbook assumes that readers are already familiar with the R language and in Part~I that they are familiar with Physics and Mathematics, including calculus and geometry.

\section{Typographical conventions}

Code examples are typeset in monospaced font and syntax highlighted in colour. References to R language elements---i.e.\ R `code'---in the main text are also in a \code{monospaced} font but in black on a faint background. Package names are typeset between single quotes in a \pkg{sans serif} font.

We\Attention{} use the icon exemplified in the page margin next to this paragraph to highlight contents that require special attention because they are frequent causes of errors and problems.

We\Advanced{} use the icon exemplified in the page margin next to this paragraph to highlight contents that is advanced and will require the reader to linger on it to get a deep understanding---and which can, alternatively, be skipped on first reading by those readers which want a faster path to learning to do simpler calculations.

\section{Acknowledgements}

We thank Stefano Catola, Paula Salonen, David Israel, Neha Rai, Tendry Randriamanana, Saara Hartikainen, Christian Bianchi-Str{\o}mme, Fang Wang and \ldots for very useful comments and suggestions on the draft manuscript and examples used in training schools. The friendly and generous R community also deserves a big `Thank you!'.

Helsinki, August 2016.

The authors. 