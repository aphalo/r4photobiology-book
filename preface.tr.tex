\chapter{Preface}

This is the draft of a handbook that accompanies the release of the suite of R packages for photobiology (\textsf{r4photobiology}). Although this text includes many different recipes, it is not comprehensive in covering all the functionality of the packages. The organization of the cookbook (Part III) is according to tasks. The packages themselves include \emph{User Guides} and help pages describing the functionality in detail.

This handbook assumes that readers are already familiar with the R language. The biophysical theory needed to follow the examples is concisely described in Part I, but the treatment is very concise.

\section{Typographical conventions}

Code examples are typeset in monospaced font and syntax highlighted in colour. References to R language elements---i.e.\ R `code'---in the main text are also in a \code{monospaced} font but in black on a faint background. Package names are typeset between single quotes in a \pckg{sans serif} font.

We\Attention{} use the icon exemplified in the page margin next to this paragraph to highlight contents that require special attention because they are frequent causes of errors and problems.

We\Advanced{} use the icon exemplified in the page margin next to this paragraph to highlight contents that is advanced and will require the reader to linger on it to get a deep understanding when reading---which can, alternatively, be skipped on first reading by those readers which want a faster path to learning to do simpler calculations.

\section{Acknowledgements}

We thank Stefano Catola, Paula Salonen, David Israel, Neha Rai, Tendry Randriamanana, Saara Harkikainen, Christian Bianchi-Str{\o}mme and \ldots for very useful comments and suggestions on the draft manuscript and examples used in training schools.
