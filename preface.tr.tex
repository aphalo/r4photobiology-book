\chapter{Preface}

This is the draft of a handbook that accompanies the release of the suite of R packages for photobiology (\textsf{r4photobiology}). The biophysical theory needed to follow the examples is concisely described in Part~I, but the treatment is very concise\autocite[][is an exhaustive treaty on phototobiology]{Bjoern2015}. Methods for research in the UV photobiology of plants \autocite{Aphalo2012} covers theory and practice of experimentation in photobiology. The software used in the examples including a suite of R packages developed by one of the authors is described in Part~II. A cookbook with recipes for different calculation tasks is in Part~III. Part~IV includes three chapters dedicated to reading and writing data in \emph{foreign} formats, including direct acquisition from measuring instruments. Part~V gives a brief overview of the data sets included in the suite. Although this text includes many different recipes, it is not comprehensive in covering all the functionality of the packages. The packages themselves include \emph{User Guides} and help pages describing their functionality in detail. A series of articles describing specific aspects of the use of the suite is being published in the \href{http://uv4plants.org/publications/uv4plants-bulletin-published-issues/}{UV4Plants Bulletin}.

This handbook assumes that readers are already familiar with the R language and in Part~I that they are familiar with Physics and Mathematics, including calculus and geometry.

\section{Typographical conventions}

Code examples are typeset in monospaced font and syntax highlighted in colour. References to R language elements---i.e.\ R `code'---in the main text are also in a \code{monospaced} font but in black on a faint background. Package names are typeset between single quotes in a \pckg{sans serif} font.

We\Attention{} use the icon exemplified in the page margin next to this paragraph to highlight contents that require special attention because they are frequent causes of errors and problems.

We\Advanced{} use the icon exemplified in the page margin next to this paragraph to highlight contents that is advanced and will require the reader to linger on it to get a deep understanding---and which can, alternatively, be skipped on first reading by those readers which want a faster path to learning to do simpler calculations.

\section{Acknowledgements}

We thank Stefano Catola, Paula Salonen, David Israel, Neha Rai, Tendry Randriamanana, Saara Harkikainen, Christian Bianchi-Str{\o}mme and \ldots for very useful comments and suggestions on the draft manuscript and examples used in training schools. The friendly and generous R community also deserves a big `Thank you!'.  

Helsinki, July 2016.

The authors.