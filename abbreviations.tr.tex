% -*- TeX -*- -*- UK -*-
%%
%% This file is included in `Good_Practice.tex'.
%%

\chapter{List of abbreviations and symbols}

For quantities and units used in photobiology we follow, as much as possible, the recommendations of the Commission Internationale de l'Éclairage as described by \autocite{Sliney2007}.
\begin{tabbing}
\makebox[6em][l]{\textbf{Symbol}} \= \textbf{Definition} \\[1ex]
\abst  \> \gls{absorptance} (\%).\\
\deltae \> water vapour pressure difference (Pa).\\
\emitt \> emittance (\watt).\\
$\lambda$ \>  wavelength (nm).\\
\SZA   \> solar zenith angle (degrees).\\
$\nu$ \> frequency (Hz or s$^{-1}$).\\
\refl  \> \gls{reflectance} (\%).\\
$\sigma$ \> Stefan-Boltzmann constant.\\
\trans \> \gls{transmittance} (\%).\\
$\chi$ \> water vapour content in the air (\gmcubic).\\
\absb  \> \gls{absorbance} (absorbance units).\\
ANCOVA \> analysis of covariance.\\
ANOVA \> analysis of variance.\\
BSWF \> \gls{BSWF}.\\
$c$ \> speed of light in a vacuum.\\
CCD \> charge coupled device, a type of light detector.\\
CDOM \> coloured dissolved organic matter.\\
CFC \> chlorofluorocarbons.\\
c.i.\> confidence interval.\\
CIE \>  Commission Internationale de l'Éclairage;\\% (International Commission on Illumination);\\
    \>  or erythemal action spectrum standardized by CIE.\\
CTC \> closed-top chamber.\\
DAD \> diode array detector, linear light detector based on photodiodes.\\
DBP \> dibutylphthalate.\\
DC \> direct current.\\
DIBP \> diisobutylphthalate.\\
DNA(N) \> \UV action spectrum for `naked' DNA.\\
DNA(P) \> \UV action spectrum for DNA in plants.\\
DOM \> dissolved organic matter.\\
DU \> Dobson units.\\
$e$ \> water vapour partial pressure (Pa).\\
\irr \> (energy) irradiance (\watt).\\
\sirr \> spectral (energy) irradiance (\wattnm).\\
\flrat \> fluence rate, also called scalar irradiance (\watt).\\
%EHWS \> extreme high water-level spring tides (m).\\
%ELWS \> extreme low water-level  spring tides (m).\\
ESR \> early stage researcher.\\
FACE \> free air carbon-dioxide enhancement.\\
FEL \> a certain type of 1000~W incandescent lamp.\\
FLAV \> \UV action spectrum for accumulation of flavonoids.\\
FWHM \> full-width half-maximum.\\
GAW \> Global Atmosphere Watch.\\
GEN \> generalized plant action spectrum, also abreviated as GPAS \autocite{Caldwell1971}.\\
GEN(G) \> mathematical formulation of GEN by \autocite{Green1974} .\\
GEN(T) \> mathematical formulation of GEN by \autocite{Thimijan1978}.\\
$h$ \> Planck's constant.\\
$h^\prime$ \> Planck's constant per mole of photons.\\
\exposure \> exposure, frequently called dose by biologists (\kjday).\\
\dose[BE]  \> biologically effective (energy) exposure (\kjday).\\
%\sdose
\qdose[BE] \> biologically effective photon exposure (\molday).\\
HPS \> high pressure sodium, a type of discharge lamp.\\
HSD \> honestly signifcant difference.\\
$k_\mathrm{B}$ \> Boltzmann constant.\\
\rad   \> radiance (\wattsr).\\
LAI \> leaf area index, the ratio of projected leaf area to the ground area.\\
LED \> light emitting diode.\\
LME \> linear mixed effects (type of statistical model).\\
LSD \> least significant difference.\\
$n$ \> number of replicates (number of experimental units per treatment).\\
$N$ \> total number of experimental units in an experiment.\\
$N_\mathrm{A}$ \> Avogadro constant (also called Avogadro's number).\\
NIST \> National Institute of Standards and Technology (U.S.A.).\\
NLME \> non-linear mixed effects (statistical model).\\
OTC \> open-top chamber.\\
\PAR  \> \gls{PAR}, 400--700~nm.\\
      \> measured as energy or photon irradiance.\\
PC \> polycarbonate, a plastic.\\
PG \> \UV action spectrum for plant growth.\\
PHIN \> \UV action spectrum for photoinhibition of isolated chloroplasts.\\
PID \> \gls{PID} (control algorithm).\\
PMMA  \> polymethylmethacrylate.\\
\PPFD \> \gls{PPFD}, another name for\\
      \> \PAR photon irradiance (\pfd[PAR]).\\
PTFE  \> polytetrafluoroethylene.\\
PVC  \> polyvinylchloride.\\
$q$ \> energy in one photon (`energy of light').\\
$q^\prime$ \> energy in one mole of photons.\\
\pfd  \> photon irradiance (\molms or \umol).\\
\spfd \> spectral photon irradiance (\molnm or \umolnm).\\
r$_0$ \> distance from sun to earth.\\
\RAF \> \gls{RAF} (nondimensional).\\
RH \> relative humidity (\%).\\
\eeff   \> energy effectiveness (relative units).\\
\seeff  \> spectral energy effectiveness (relative units).\\
\qeff   \> quantum effectiveness (relative units).\\
\sqeff  \> spectral quantum effectiveness (relative units).\\
s.d. \> standard deviation.\\
SDK \> software development kit.\\
s.e. \> standard error of the mean.\\
SR \> spectroradiometer.\\
%\sqdose
$t$ \> time.\\
$T$ \> temperature.\\
TUV \> tropospheric \UV.\\
\voltage  \> electric potential difference or voltage (e.g.\ sensor output in V).\\
\UV \> ultraviolet radiation ($\lambda =$ 100--400 nm).\\
\UVA \> ultraviolet-A radiation ($\lambda =$ 315--400 nm).\\
\UVB \> ultraviolet-B radiation ($\lambda =$ 280--315 nm).\\
\UVC \> ultraviolet-C radiation ($\lambda =$ 100--280 nm). \\
\UVeff \> biologically effective \UV radiation.\\
UTC \> coordinated universal time, replaces GMT in technical use.\\
VIS \> radiation visible to the human eye ($\approx$ 400--700~nm).\\
WMO \> World Meteorological Organization.\\
VPD \> water vapour pressure deficit (Pa).\\
WOUDC \> World Ozone and Ultraviolet Radiation Data Centre.\\
\end{tabbing}
